\documentclass{article}

\usepackage{microtype}
\usepackage{listings}
\usepackage{tabularx}
\usepackage{float}
\usepackage{gensymb}

\lstset{
  basicstyle=\ttfamily,
  columns=fullflexible,
  frame=single,
  breaklines=true,
  language=python,
  tabsize=4
}

\title{Advanced Trees}
\author{Jakob Wyatt\\19477143}

\begin{document}
\maketitle
\section{Height Comparison}

\begin{table}[H]
\centering
\begin{tabularx}{\linewidth}{| c | X | X | X |}
\hline
 & \textbf{Randomized (Size 9)} & \textbf{Increasing (Size 8)} & \textbf{Decreasing (Size 11)} \\
\hline
\textbf{Binary Search Tree} & 4 & 7 & 10 \\
\hline
\textbf{2-3-4 Tree} & 1 & 1 & 2 \\
\hline
\textbf{B-Tree (Degree 5)}  & 1 & 1 & 1 \\
\hline
\textbf{Red-Black Tree} & 3 & 3 & 4 \\
\hline
\end{tabularx}
\end{table}

The binary search tree becomes degenerate for the increasing and decreasing datasets.
The other trees avoid this, as they are self balancing. The 2-3-4 trees and B-Tree
will often have a lower height than the red-black tree, as they contain more keys per node.

\section{Complexity Analysis}

All self balancing trees are given as average-case complexity.

\begin{table}[H]
\centering
\begin{tabularx}{\linewidth}{| X | X | X | X |}
\hline
 & \textbf{Insert} & \textbf{Find} & \textbf{Delete} \\
\hline
\textbf{Binary Search Tree (Average Case)} & $O\left(\log_2n\right)$ & $O\left(\log_2n\right)$ & $O\left(\log_2n\right)$ \\
\hline
\textbf{Binary Search Tree (Worst Case)} & $O\left(n\right)$ & $O\left(n\right)$ & $O\left(n\right)$ \\
\hline
\textbf{2-3-4 Tree} & $O\left(\log_4n\right)$ & $O\left(\log_4n\right)$ & $O\left(\log_4n\right)$ \\
\hline
\textbf{B-Tree (Degree $d$)}  & $O\left(d\log_dn\right)$ & $O\left(d\log_dn\right)$ & $O\left(d\log_dn\right)$ \\
\hline
\textbf{Red-Black Tree} & $O\left(\log_2n\right)$ & $O\left(\log_2n\right)$ & $O\left(\log_2n\right)$ \\
\hline
\end{tabularx}
\end{table}

The worst case time complexities were not included for the self-balancing binary trees,
as they do not degrade as badly as the binary search tree does.
The time complexity of the 2-3-4 tree initially looks better than the red-black tree. However,
due to the large constant factors contained in having 3 keys per node, it can often
end up being slower in implementation.
The self balancing trees will have much better performance than the BST on degenerate datasets,
as the automatic balancing allows them to retain performance on the order of $O\left(\log n\right)$.

\section{Difficulty of Implementation}
I believe that the easiest tree to implement
would be the 2-3-4 tree, as it has a fixed size
and an easily understandable node splitting
strategy.
Next would be the B-Tree, as it is a more
generalized version of the 2-3-4 tree, however
still has an easily understandable splitting
strategy / algorithm.
I believe that the most difficult tree to implement would
be the red-black tree, as the insertion algorithm has many special
cases to consider, which adds more room for error.
\section{In-Order Traversal}
An in order traversal on a red-black tree is the same as the traversal
of a normal binary search tree.
An in order traversal of a B-tree could be implemented using recursion,
using the example code below.
\begin{lstlisting}
def inOrderBTree(node):
    # len(node.key) + 1 == len(node.children)
    for n, c in zip(node.key, node.children[:-1]):
        inOrderBTree(c)
        process(n)
    inOrderBTree(node.children[-1])
\end{lstlisting}
An in order traversal on a 2-3-4 tree is a special case of
the traversal of a B-tree of order 4.
\end{document}
